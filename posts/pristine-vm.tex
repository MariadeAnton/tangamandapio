\documentclass[12pt]{article}

\usepackage{graphicx}

\title{Clone virtual machines to keep a pristine image}

\begin{document}

The basic idea of using virtual machines for testing is to always start from a
clean state, as close as possible to what a user would get after freshly
installing the system on their real machines.

So, every time we need to test something, we could
\href{http://elopio.net/blog/install-ubuntu-in-vm/}
     {create a new virtual machine and intall Ubuntu from scratch there},
but that would take a considerable amount of time.

An alternative is to install a machine once, and keep it clean. Never test in
that one, and instead clone it every time you need to test something. Use the
clone to experiment, and delete it when you are done. Cloning the machine is
a lot faster than reinstalling each time. The base machine, the one you
clone everytime, is called a pristine machine. If you are careful with it, the
only time you will need to reinstall is when you are testing the installer.

Cloning the pristine machine using Virtual Machine Manager is really simple,
it's done in like two clicks.

First, of course, you will need to
\href{http://elopio.net/blog/install-ubuntu-in-vm/}
     {install your pristine machine}.
I always put pristine in the name, to make it less likely that I will start
using it for testing. Every time I forget and play with the pristine machine,
I'm polluting its state, and have to recreate it.

With the pristine machine ready, I then recommend to open it and run in a
terminal:

\begin{verbatim}
  sudo apt update && sudo apt upgrade
\end{verbatim}

That will update all the packages installed, so your clone starts also in the
most up-to-date state. If you do this step every time, you will never have to
wait for long while your machine downloads and installs lots of packages, just
wait a little every day.

\begin{center}
  \href{
    https://archive.org/download/elopio-screenshots2/clone/1-update.png}{
    \includegraphics[width=500px]{
    https://archive.org/download/elopio-screenshots2/clone/1-update.png}
  }
  \caption{Update the pristine virtual machine}
\end{center}

Now shut down the pristine machine and don't touch it again, only to update it.

To clone it, right click on it and then click the \emph{Clone...} button.

\begin{center}
  \href{
    https://archive.org/download/elopio-screenshots2/clone/2-clone.png}{
    \includegraphics[width=500px]{
    https://archive.org/download/elopio-screenshots2/clone/2-clone.png}
  }
  \caption{Clone a machine}
\end{center}

A dialog will be opened, where you can enter the new virtual machine details.

\begin{center}
  \href{
    https://archive.org/download/elopio-screenshots2/clone/3-cloneinfo.png}{
    \includegraphics[width=500px]{
    https://archive.org/download/elopio-screenshots2/clone/3-cloneinfo.png}
  }
  \caption{Clone dialog}
\end{center}

I always write the date as part of the name, to remember when I created it.
But the most important thing to do here is to make sure that the \emph{Storage}
option is set to \emph{Clone this disk}. Otherwise, you will be sharing the disk
with the pristine machine, which will pollute its state, the exact thing we are
trying to avoid.

This will take some time while the whole disk is copied. But not long, and
once it's done you can freely play and pollute this clone, without affecting the
pristine machine.

\end{document}
