\documentclass[12pt]{article}

\usepackage{graphicx}

\title{Download the latest Ubuntu LTS image}

\begin{document}

The \href{https://wiki.ubuntu.com/LTS}{Ubuntu Long Term Support} (LTS) version
is released every two years, on April; and it is supported for five years. So
this version has to be super stable, it's what most of our users have installed
in their machines, and it often gets security fixes and some other
\hrev{https://wiki.ubuntu.com/StableReleaseUpdates}{stable release updates}.

All of this means that it always needs to be tested, and by as many hands and on
as many different machines as possible. If you want to start helping the Ubuntu
community, I would suggest that testing the latest LTS is the place to start.
There will be so many people around the world that will be positively affected
by this work, and all of them will sincerely appreciate it. Well... not all of
them, but most of them for sure :) I for one will invite you to a beer when you
come for holidays to Costa Rica.

While testing Ubuntu you will learn about the many different aspects that build
a free operating system. Then you can jump to collaborate in other areas that
pique your interest, like support, programming, translations, packaging,
planning, partying, etc. Or you can stay helping with the tests, maybe in some
more unstable and rapidly changing projects.

This is the first post of a series I will write about contributing to different
areas of Ubuntu. They will focus mostly on testing because that's what I enjoy
the most. And in addition, we will meet live every Friday on
\href{http://ubuntuonair.com/}{Ubuntu On-Air} to explore cool free software
projects and help the developers on their way to a stable release.

Let's begin. The first thing you will have to do if you want to join our
community is to download Ubuntu. To do that open your browser and go to
https://www.ubuntu.com/download

Click the \emph{Ubuntu Desktop} link, or the download button next to it.

\begin{center}
  \href{
    https://ia600208.us.archive.org/24/items/elopio-screenshots/install-ubuntu/1-download-page.png}{
    \includegraphics[width=500px]{
    https://ia600208.us.archive.org/24/items/elopio-screenshots/install-ubuntu/1-download-page.png}
  }
  \caption{Ubuntu Download page}
\end{center}

There are other different versions of Ubuntu. Newer and older releases,
different user interface flavors, smaller versions for servers and connected
devices, and some for specific phones and tablets. We will see some of them
in following posts and meetings, but for now we will start with the basic and
most common.

The next step shows you the recommended requirements that a machine should
meet to run this Ubuntu version properly. Make sure that you have all of them
and click the \emph{Download} button.

\begin{center}
  \href{
    https://ia600208.us.archive.org/24/items/elopio-screenshots/install-ubuntu/2-requirements.png}{
    \includegraphics[width=500px]{
    https://ia600208.us.archive.org/24/items/elopio-screenshots/install-ubuntu/2-requirements.png}
  }
  \caption{Ubuntu Requirements page}
\end{center}

This opens a page for optional donations. If you feel generous, you can give
some money to the community and decide where it should be spent. If you can't
or don't want to give any money, no problem, just click the link that says:
\emph{Not now, take me to the download}. Beers, hugs, bugs are all valid
payment methods too.

\begin{center}
  \href{
    https://ia600208.us.archive.org/24/items/elopio-screenshots/install-ubuntu/3-donations.png}{
    \includegraphics[width=500px]{
    https://ia600208.us.archive.org/24/items/elopio-screenshots/install-ubuntu/3-donations.png}
  }
  \caption{Ubuntu Donations page}
\end{center}

And that's it. A dialog should appear offering you to download the file. Make
sure to choose the option to save it into your disk, and take into account
that if you use a different browser than mine, your dialog might look
different.

\begin{center}
  \href{
    https://ia600208.us.archive.org/24/items/elopio-screenshots/install-ubuntu/4-save-file.png}{
    \includegraphics[width=500px]{
    https://ia600208.us.archive.org/24/items/elopio-screenshots/install-ubuntu/4-save-file.png}
  }
  \caption{Download dialog}
\end{center}

This was a first simple step. If it was too simple for you, don't worry, next
time we will do a few more complicated things. If on the other hand you got
lost somewhere, also don't worry because we will be around to answer any kind
of doubts.

In both cases, come and join us next Friday, November 25th at 18:00 UTC on our
first Ubuntu Testing Day! We will use this image to test the awesome
\href{https://nextcloud.com/}{Nextcloud} snap.
\href{https://kyrofa.com/}{Kyle Fazarri} has been making some great work to
package Nextcloud for everybody, and he will be joining us to show off and
answer questions.

\end{document}
