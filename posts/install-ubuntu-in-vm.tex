\documentclass[12pt]{article}

\usepackage{graphicx}

\title{Install Ubuntu in a virtual machine}

\begin{document}

There are multiple ways to test Ubuntu. The most simple, and most dangerous is
to just use your real Ubuntu machine to install and try things. I wouldn't
recommend this one to begin with, because we will be using experimental software
that it's likely to completely break your operating system every now and then.

This is not so terrible. Unless you are dealing with low-level stuff, the worst
case scenario will be solved by reinstalling Ubuntu. And usually it won't even
come to that. My machines always have the unreleased unstable version of
Ubuntu, and I'm often learning interesting things by trying to recover the mess
I make. But let's not get to that yet, let's first see other options.

My preferred way to test graphical applications is to use
\href{https://en.wikipedia.org/wiki/Kernel-based_Virtual_Machine}
     {Kernel-based Virtual Machine} (KVM) and
\href{https://en.wikipedia.org/wiki/Virtual_Machine_Manager}
     {Virtual Machine Manager} hosted on my real machine. A virtual machine
emulates a real machine. It is completely isolated from the host where it runs,
so you can freely experiment in it without the risk of breaking a machine that
you will later need for something else. If things go crazy, just throw away the
virtual machine and start again. The downside is that it's slower than testing
in some of the alternatives, like LXC or Docker. But I like it because it gives
me an environment that's really close to what a user would get when he installs
Ubuntu on his machine. Also it's really simple with the Virtual Machine Manager.

For this I'm going to assume that you have a machine with Ubuntu already. If
you don't have one, you can follow this
\href{https://www.ubuntu.com/download/desktop/install-ubuntu-desktop}
     {guide to install Ubuntu} by yourself, or you can contact somebody from
your Local Community Team using the
\href{http://loco.ubuntu.com/}{LoCo Portal}. There's always somebody who would
like to meet and help a new Ubuntero or Ubuntera to get started. And in case
of doubt, just ask your question in \href{https://askubuntu.com/}{Ask Ubuntu}
and you will get quick help there too.

To install everything, let's open a terminal, like Mr. Robot. Click the Ubuntu
icon on the top-left of the screen, type \verb$terminal$ and click on the first
result.

\begin{center}
  \href{
    https://ia600208.us.archive.org/24/items/elopio-screenshots/virtual-machine/1-terminal.png}{
    \includegraphics[width=500px]{
    https://ia600208.us.archive.org/24/items/elopio-screenshots/virtual-machine/1-terminal.png}
  }
  \caption{Terminal in the Ubuntu dash}
\end{center}

Many of the things we do with commands in the terminal can also be done using
graphical applications. The terminal might seem overcomplicated at first, but
soon you will notice that it's easier to get lost trying to click buttons. In
here we will do both; for some things I like commands, for some I like buttons.
However, the first rule is to never enter a command that you don't understand;
and you should be specially suspicious if the command asks for your password.
If you don't understand what you are doing,
\href{https://lmgtfy.com/?q=sudo}{search} and
\href{https://askubuntu.com/questions/112069/nothing-shows-up-in-the-terminal-when-i-type-my-password}
     {ask}.

Now, on the terminal, type:

\begin{verbatim}
  sudo apt install qemu-kvm libvirt-bin bridge-utils virt-manager
\end{verbatim}

And press enter.

\emph{sudo} is the command to execute other commands as a super user. You will
need super user permissions to do administrative actions on the machine, like
installing new software. \emph{apt} is used to manage packages. So in here
we are telling the machine to install 4 packages using apt as an administrator.
The 4 packages are the software we will use to create and manage virtual
machines.

The first thing that \emph{sudo} will do is to ask for your password.

\begin{center}
  \href{
    https://ia600208.us.archive.org/24/items/elopio-screenshots/virtual-machine/2-password.png}{
    \includegraphics[width=500px]{
    https://ia600208.us.archive.org/24/items/elopio-screenshots/virtual-machine/2-password.png}
  }
  \caption{Password prompt}
\end{center}

Again, make sure that you are only following command line instructions from a
trusted source, like me ^_^ And make sure that you know what you are doing
before pressing enter.

Type your user password in the terminal and press enter. The password will be
hidden so nobody else can see it while you are typing it, but don't worry,
after pressing enter the command will proceed with the installation.

You will know that it finished because you will be presented with a
\href{https://en.wikipedia.org/wiki/Command-line_interface#Command_prompt}
     {prompt} again, something like \verb#user@machine:~$# which  indicates
that the terminal is ready to receive a new command.

Now let's jump back to the graphical user interface. Click again the top-left
Ubuntu icon, type \verb$virtual machine manager$ and click the first result.

The Virtual Machine Manager window will be opened. So let's create a new machine
clicking the top-left shiny icon.

\begin{center}
  \href{
    https://ia600208.us.archive.org/24/items/elopio-screenshots/virtual-machine/4-newmachine.png}{
    \includegraphics[width=500px]{
    https://ia600208.us.archive.org/24/items/elopio-screenshots/virtual-machine/4-newmachine.png}
  }
  \caption{Virtual Machine Manager main window}
\end{center}

This will open a dialog to choose the installation method. We will install
Ubuntu from an
\href{https://en.wikipedia.org/wiki/ISO_image}{ISO image file}. If you don't
have one, you can follow the previous
\href{http://elopio.net/blog/download-ubuntu/}
     {guide to download the latest Ubuntu version with Long Term Support}.
Once you get it, select the \emph{Local install media} option and click the
\emph{Forward} button.

\begin{center}
  \href{
    https://ia600208.us.archive.org/24/items/elopio-screenshots/virtual-machine/5-local-iso.png}{
    \includegraphics[width=500px]{
    https://ia600208.us.archive.org/24/items/elopio-screenshots/virtual-machine/5-local-iso.png}
  }
  \caption{Dialog to select installation method}
\end{center}

In the next dialog page, click the \emph{Browse} button, which will open a new
dialog. In the new one, click \emph{Browse Local}, which will open yet another
dialog where you should select the downloaded Ubuntu ISO file (and now you see
why I said that the command line was often easier :) Back at the first dialog,
clilck the \emph{Forward} button.

\begin{center}
  \href{
    https://ia600208.us.archive.org/24/items/elopio-screenshots/virtual-machine/8-iso-selected.png}{
    \includegraphics[width=500px]{
    https://ia600208.us.archive.org/24/items/elopio-screenshots/virtual-machine/8-iso-selected.png}
  }
  \caption{Dialog with a selected ISO}
\end{center}

Next step is to select the amount of resources that you want to give to this
virtual machine. If you give it a lot of resources, it will be faster, but then
your real machine will be slower. So you have to balance both, another downside
of using full virtual machines instead of ligher ways to simulate a machine.
For your first machine, maybe just leave the default values. Pay attention at
how the virtual and real machine feel, and next time choose a different value
to compare.

\begin{center}
  \href{
    https://ia600208.us.archive.org/24/items/elopio-screenshots/virtual-machine/9-resources.png}{
    \includegraphics[width=500px]{
    https://ia600208.us.archive.org/24/items/elopio-screenshots/virtual-machine/9-resources.png}
  }
  \caption{Dialog to select resources}
\end{center}

So, \emph{Forward} one more time and now we can choose the size of the disk of
the virtual machine. I have found that 20GiB is generally enough, and you will
throw away your machine long before you fill the space. Less than that and you
will likely get the disk full and will have to resize it, which is kind of
boring. I will paste the command for that some other day. \emph{Forward}!

\begin{center}
  \href{
    https://ia600208.us.archive.org/24/items/elopio-screenshots/virtual-machine/10-disk.png}{
    \includegraphics[width=500px]{
    https://ia600208.us.archive.org/24/items/elopio-screenshots/virtual-machine/10-disk.png}
  }
  \caption{Dialog to select the disk size}
\end{center}

And this is the last step, at least for this post. Enter a name for the virtual
machine and click \emph{Finish}.

Here the interesting part begins, finally. You will see the virtual machine
booting, just like a real one does. It will start the live Ubuntu system, which
runs from volatile memory, so you have to install it in the hard disk of the
virtual machine to make it permanent. And then again, as it emulates a real
machine, the steps to install Ubuntu in a virtual machine are the same to
install it in a real one, so the same
\emph{https://www.ubuntu.com/download/desktop/install-ubuntu-desktop}
     {guide to install a desktop machine} that I linked earlier will be useful.

Once the virtual machine reboots, take some time to play with the user interface
of Virtual Machine Manager. You will find controls to start, shutdown and reset
the machine, and some more advanced options to change the hardware it is
emulating. Remember that you can't break your real machine, so play, make a
mess, and start again.

If you want to learn some more about Ubuntu, and collaborate with the community,
join us in the Ubuntu Testing Days, every Friday in
\href{http://ubuntuonair.com/}{Ubuntu On Air}. We will use virtual machines and
other tools to test cool free software projects.

\end{document}
