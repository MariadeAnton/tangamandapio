\documentclass[12pt]{article}

\usepackage{cite}
\usepackage{graphicx}

\title{Sharing the Wi-Fi connection through Ethernet}

\begin{document}

My apartment was clearly not designed for the Internet of Things. I have a
weird set up to connect all my devices: there's ADSL, Ethernet, Wi-Fi,
powerline networking, 4G from a free promotion until I download 3GB and then
3G for the rest of the month... It's crazy, and getting crazier as I have to
test more and more devices for
\href{https://developer.ubuntu.com/en/snappy/start/}{snappy}. When I was about
to throw a cable from the kitchen to the office I found about
\href{http://askubuntu.com/a/174027/61416}{sharing the network} (thanks
\href{https://askubuntu.com/users/7035/luis-alvarado}{Luis}!). Now let me
repeat a slightly modified version of part of Luis' answer to show how I
share my wireless connection through an ethernet cable with my test board.

\begin{enumerate}
  \item Open the network indicator.
  \item Click \emph{Edit Connections...}.
    \begin{center}
      \href{
        https://ia601505.us.archive.org/25/items/elopio-screenshots/shared1.png}{
        \includegraphics[width=500px]{
          https://ia601505.us.archive.org/25/items/elopio-screenshots/shared1.png}
      }
      \caption{Network indicator menu}
    \end{center}
  \item On the \emph{Network Connections} dialog, click the \emph{Add} button.
    \begin{center}
      \href{
        https://ia601505.us.archive.org/25/items/elopio-screenshots/shared2.png}{
        \includegraphics[width=500px]{
          https://ia601505.us.archive.org/25/items/elopio-screenshots/shared2.png}
      }
      \caption{Network Connections dialog}
    \end{center}
  \item On the \emph{Choose a Connection Type} dialog, select \emph{Ethernet}.
    \begin{center}
      \href{
        https://ia601505.us.archive.org/25/items/elopio-screenshots/shared3.png}{
        \includegraphics[width=500px]{
          https://ia601505.us.archive.org/25/items/elopio-screenshots/shared3.png}
      }
      \caption{Choose a Connection Type dialog}
    \end{center}
  \item On the \emph{Editing} dialog, enter a name for the connection.
  \item Go to the \emph{IPv4 Settings} tab.
  \item Select the Method \emph{Shared to other computers}.
    \begin{center}
      \href{
        https://ia601505.us.archive.org/25/items/elopio-screenshots/shared4.png}{
        \includegraphics[width=500px]{
          https://ia601505.us.archive.org/25/items/elopio-screenshots/shared4.png}
      }
      \caption{Editing dialog}
    \end{center}
  \item Click the \emph{Save} button.
\end{enumerate}

What's left is to connect an Ethernet cable from your laptop to the board,
give power to the board and wait for it to finish booting.

To get the IP of the board you can run the \verb$arp$ command (Thanks to
Alex for the tip). It will show you the addresses of the neighbour machines.
The one of your board will be like \verb$10.42.0.?$. Now you can ssh into the
board using the default credentials (username \verb$ubuntu$, password
\verb$ubuntu$) or the ones you configured.

\begin{center}
  \href{
    https://ia601505.us.archive.org/33/items/JaquerEspeis-bbb-snappy/image20160302_194108865.jpg}{
    \includegraphics[width=500px]{
      https://ia601505.us.archive.org/33/items/JaquerEspeis-bbb-snappy/image20160302_194108865.jpg}
  }
  \caption{BeagleBone connected to a laptop through the Ethernet cable}
\end{center}

This also proved to be useful during our
\href{https://archive.org/details/JaquerEspeis-bbb-snappy}{first snappy maker
night}, when we had to connect many boards to play with the system.

Pro tip: If you are having problems connecting through \verb$ssh$, you can
connect
\href{http://elopio.net/blog/connecting-to-snappy-through-the-serial-console/}
{through the serial console} to check for errors.

\end{document}
